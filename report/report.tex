%%%%%%%%%%%%%%%%%%%%%%%%%%%%%%%%%%%%%%%%%
% University/School Laboratory Report
% LaTeX Template
% Version 3.1 (25/3/14)
%
% This template has been downloaded from:
% http://www.LaTeXTemplates.com
%
% Original author:
% Linux and Unix Users Group at Virginia Tech Wiki 
% (https://vtluug.org/wiki/Example_LaTeX_chem_lab_report)
%
% License:
% CC BY-NC-SA 3.0 (http://creativecommons.org/licenses/by-nc-sa/3.0/)
%
%%%%%%%%%%%%%%%%%%%%%%%%%%%%%%%%%%%%%%%%%

%----------------------------------------------------------------------------------------
%	PACKAGES AND DOCUMENT CONFIGURATIONS
%----------------------------------------------------------------------------------------

\documentclass{article}

%\usepackage[version=3]{mhchem} % Package for chemical equation typesetting
%\usepackage{siunitx} % Provides the \SI{}{} and \si{} command for typesetting SI units
\usepackage{graphicx} % Required for the inclusion of images
%\usepackage{natbib} % Required to change bibliography style to APA
%\usepackage{amsmath} % Required for some math elements 
\usepackage{listings}
\lstset{
  breaklines=true,
  basicstyle=\scriptsize,
  columns=fullflexible
}

\usepackage{tikz}

\setlength\parindent{0pt} % Removes all indentation from paragraphs

\renewcommand{\labelenumi}{\alph{enumi}.} % Make numbering in the enumerate environment by letter rather than number (e.g. section 6)

%\usepackage{times} % Uncomment to use the Times New Roman font

%----------------------------------------------------------------------------------------
%	DOCUMENT INFORMATION
%----------------------------------------------------------------------------------------

\title{Report: Homework 5 reloaded - Advanced Programming}% Title

\author{Jan \textsc{Schlenker}} % Author name

\date{\today} % Date for the report

\begin{document}

\maketitle % Insert the title, author and date

\begin{center}
\begin{tabular}{l l}
Instructor: & Dipl.-Ing. Dr. Simon Ostermann \\
Parts solved of the sheet: & Grid task \\
Programming language: & Java \\
Library used: & JGlobus (https://github.com/jglobus/JGlobus) \\
Total points: & 15 \\
\end{tabular}
\end{center}

% If you wish to include an abstract, uncomment the lines below
% \begin{abstract}
% Abstract text
% \end{abstract}

%----------------------------------------------------------------------------------------
%	SECTION 1
%----------------------------------------------------------------------------------------
\section{Requirements}

\begin{itemize}
\item Java 1.7
\item Ant 1.9.2
\end{itemize}

\section{How to run the programme}

First of all extract the archive file \texttt{homework\_5\_2.tar.gz}:

\begin{lstlisting}[language=bash, deletekeywords={cd}]
  $ tar -xzf homework_5_2.tar.gz
  $ cd homework_5_2
\end{lstlisting}

Afterwards move/copy the povray files \texttt{povray}, \texttt{gm} and \texttt{scherk.pov} to the \texttt{povray/} directory:

\begin{lstlisting}[language=bash]
  $ cp <gm-file-path> <povray-file-path> <scherk.pov-file-path> povray/
\end{lstlisting}

Now use ant to compile the sources and build the runnable jar:

\begin{lstlisting}[language=bash]
  $ ant compile jar
\end{lstlisting}

If you have no proxy, initialize one:

\begin{lstlisting}[language=bash]
  $ grid-proxy-init
\end{lstlisting}

At last run the created jar with ant, where \texttt{<frames>} is the number of frames which should be rendered:

\begin{lstlisting}[language=bash]
  $ ant run -Darg0=<frames>
\end{lstlisting}

%\begin{center}\ce{2 Mg + O2 -> 2 MgO}\end{center}

% If you have more than one objective, uncomment the below:
%\begin{description}
%\item[First Objective] \hfill \\
%Objective 1 text
%\item[Second Objective] \hfill \\
%Objective 2 text
%\end{description}

%----------------------------------------------------------------------------------------
%	SECTION 2$
%----------------------------------------------------------------------------------------

\section{Programme explanation}
The files of the the programme are structured as follows:

\begin{itemize}
\item The \texttt{\textbf{src}} directory contains the source file
\item The \texttt{\textbf{povray}} directory contains the povray files \texttt{povray}, \texttt{gm} and \texttt{scherk.pov} and the shell script \texttt{povray\_render.sh}, which simply runs povray with specific parameters
\item The \texttt{\textbf{lib}} directory contains the library files of \texttt{JGlobus}
\item The \texttt{\textbf{build.xml}} file contains build information for \texttt{Ant}
\end{itemize}

The source file \texttt{src/\-grid\-Renderer/\-Grid\-Renderer\-.java} basically uses the \texttt{JGlobus} library files to copy the \texttt{povray} files to the remote nodes and render the frames on them. For parallel rendering the programme uses the standard \texttt{Thread} class of Java. The locale machine collects the rendered pictures and runs the \texttt{gm} script to produce the gif file.

\section{Remark}
I wanted to use \texttt{Maven} in combination with the newest version of \texttt{JGlobus} (http://\-mvnrepository.\-com/\-artifact/\-org.\-jglobus), so that I do not have to upload the \texttt{jar}-library files. Unfortunatelly there seems to be a problem with the \texttt{checkCRLDateValidity()} (https://\-github.\-com/\-jglobus/\-JGlobus/\-blob/\-master/\-ssl-proxies/\-src/\-main/\-java/\-org/\-globus/\-gsi/\-trust\-manager/\-CRL\-Checker.java) mentioned by Matthias Lechner in the Grid forum. That's why I just used \texttt{Ant} and a modified version of the \texttt{CRLChecker}, where I commented an exception for wrongly invalid certificates.

%----------------------------------------------------------------------------------------
%	SECTION 3
%----------------------------------------------------------------------------------------

%Because of this reaction, the required ratio is the atomic weight of magnesium: \SI{16.00}{\gram} of oxygen as experimental mass of Mg: experimental mass of oxygen or $\frac{x}{1.31}=\frac{16}{0.87}$ from which, $M_{\ce{Mg}} = 16.00 \times \frac{1.31}{0.87} = 24.1 = \SI{24}{\gram\per\mole}$ (to two significant figures).

%----------------------------------------------------------------------------------------
%	SECTION 4
%----------------------------------------------------------------------------------------

%\section{Results and Conclusions}

%The atomic weight of magnesium is concluded to be \SI{24}{\gram\per\mol}, as determined by the stoichiometry of its chemical combination with oxygen. This result is in agreement with the accepted value.

%\begin{figure}[h]
%\begin{center}
%\includegraphics[width=0.65\textwidth]{placeholder} % Include the image placeholder.png
%\caption{Figure caption.}
%\end{center}
%\end{figure}
%
%----------------------------------------------------------------------------------------
%	SECTION 5
%----------------------------------------------------------------------------------------

%\section{Discussion of Experimental Uncertainty}

%The accepted value (periodic table) is \SI{24.3}{\gram\per\mole} \cite{Smith:2012qr}. The percentage discrepancy between the accepted value and the result obtained here is 1.3\%. Because only a single measurement was made, it is not possible to calculate an estimated standard deviation.

%The most obvious source of experimental uncertainty is the limited precision of the balance. Other potential sources of experimental uncertainty are: the reaction might not be complete; if not enough time was allowed for total oxidation, less than complete oxidation of the magnesium might have, in part, reacted with nitrogen in the air (incorrect reaction); the magnesium oxide might have absorbed water from the air, and thus weigh ``too much." Because the result obtained is close to the accepted value it is possible that some of these experimental uncertainties have fortuitously cancelled one another.


%----------------------------------------------------------------------------------------
%	BIBLIOGRAPHY
%----------------------------------------------------------------------------------------

%\bibliographystyle{apalike}

%\bibliography{sample}

%----------------------------------------------------------------------------------------


\end{document}
